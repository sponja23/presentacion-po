\documentclass[8pt]{beamer}

\usepackage{commonslides}
\usepackage[compatibility=false]{caption}
\usepackage{minted}

\title{Presentación para Prueba de Oposición}
\subtitle{Paradigmas de Programación -- Práctica 1, Ejercicio 12}

\setbeamertemplate{caption}{\raggedright\insertcaption\par}

\begin{document}

\newcommand{\code}[1]{\mintinline{haskell}{#1}}

\begin{frame}
    \titlepage
\end{frame}

\section{Consigna}

\begin{frame}[fragile]{Consigna}
    Considerar el siguiente tipo, que representa a los árboles binarios:
    \begin{minted}{haskell}
        data AB a = Nil | Bin (AB a) a (AB a)
    \end{minted}

    \begin{enumerate}[I]
        \item<2-> Usando recursión explícita, definir los esquemas de recursión estructural (\code{foldAB}) y primitiva (\code{recAB}), y dar sus tipos.
        \item<3-> Definir las funciones \code{esNil}, \code{altura}, \code{cantNodos} (para \code{esNil} puede utilizarse \code{case} en lugar de \code{foldAB}).
        \item<4-> Definir la función \code{mejorSegún :: (a -> a -> Bool) -> AB a -> a} para árboles. Se recomienda definir una función auxiliar para comparar la raíz con un posible resultado de la recursión para un árbol que puede o no ser \code{Nil}.
        \item<5-> Definir la función \code{esABB :: Ord a => AB a -> Bool} que cheaquea si un árbol es un árbol binario de búsqueda. Recordar que, en un árbol binario de búsqueda, el valor de un nodo es mayor o igual que los valores que aparecen en el subárbol izquierdo y es estrictamente menor que los valores que aparecen en el subárbol derecho.
        \item<6-> Justificar la elección de los esquemas de recursión utilizados para los tres puntos anteriores.
    \end{enumerate}

\end{frame}

\section{Parte I}

\section{Parte II}

\section{Parte III}

\section{Parte IV}

\section{Parte V}

\end{document}
