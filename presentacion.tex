\documentclass[8pt]{beamer}

\usepackage{commonslides}
\usepackage[compatibility=false]{caption}
\usepackage{minted}

\title{Presentación para Prueba de Oposición}
\subtitle{Paradigmas de Programación -- Práctica 1, Ejercicio 12}

\setbeamertemplate{caption}{\raggedright\insertcaption\par}

\begin{document}

\newcommand{\code}[1]{\mintinline{haskell}{#1}}

\begin{frame}
    \titlepage
\end{frame}

\section{Consigna}

\begin{frame}[fragile]{Consigna}
    Considerar el siguiente tipo, que representa a los árboles binarios:
    \begin{minted}{haskell}
        data AB a = Nil | Bin (AB a) a (AB a)
    \end{minted}

    \begin{enumerate}[I]
        \item<2-> Usando recursión explícita, definir los esquemas de recursión estructural (\code{foldAB}) y primitiva (\code{recAB}), y dar sus tipos.
        \item<3-> Definir las funciones \code{esNil}, \code{altura}, \code{cantNodos} (para \code{esNil} puede utilizarse \code{case} en lugar de \code{foldAB}).
        \item<4-> Definir la función \code{mejorSegún :: (a -> a -> Bool) -> AB a -> a} para árboles. Se recomienda definir una función auxiliar para comparar la raíz con un posible resultado de la recursión para un árbol que puede o no ser \code{Nil}.
        \item<5-> Definir la función \code{esABB :: Ord a => AB a -> Bool} que cheaquea si un árbol es un árbol binario de búsqueda. Recordar que, en un árbol binario de búsqueda, el valor de un nodo es mayor o igual que los valores que aparecen en el subárbol izquierdo y es estrictamente menor que los valores que aparecen en el subárbol derecho.
        \item<6-> Justificar la elección de los esquemas de recursión utilizados para los tres puntos anteriores.
    \end{enumerate}

\end{frame}

\section{Parte I}

\begin{frame}{Recursión Estructural -- \texttt{foldAB}}
    \begin{block}{Tip}
        Antes de definir cualquier función, \alert{dar su tipo}.
    \end{block}

    \pause

    El tipo de la función de recursión estructural \code{foldT} para cualquier \code{T} se puede definir ``automáticamente'' en base a la definición de \code{T}:
    
    \pause

    \begin{itemize}[<+->]
        \item \code{foldT} devuelve un valor de tipo \code{r} y recibe una función \code{f} por cada constructor.
        \item Las funciones reciben los mismos argumentos que los constructores correspondientes, excepto por los \textbf{argumentos recursivos} (los de tipo \code{T}): éstos se reemplazan por resultados de la recursión (de tipo \code{r}).
    \end{itemize}
\end{frame}

\begin{frame}[fragile]{Recursión Estructural -- \texttt{foldAB} (Cont.)}
    Repasamos la definición de \code{AB}:

    \begin{minted}{haskell}
        data AB a =
            | Bin (AB a) a (AB a)
            | Nil
    \end{minted}

    \pause

    El método anterior nos dice que el tipo de \code{foldAB} debe ser:
    
    \begin{minted}[escapeinside=||]{haskell}
        foldAB ::|\pause|
            (r -> a -> r -> r) -> -- Caso Bin (AB a) a (AB a)|\pause|
            r ->                  -- Caso Nil|\pause|
            r                     -- Resultado
    \end{minted}

    \pause

    La definición de la función se desprende casi directamente:
    \begin{minted}[escapeinside=||]{haskell}
        foldAB :: (r -> a -> r -> r) -> r -> r|\pause|
        foldAB f z Nil = z|\pause|
        foldAB f z (Bin izq val der) = f (rec izq) val (rec der)|\pause|
            where rec ab = foldAB f z ab
        
    \end{minted}
\end{frame}


\section{Parte II}

\begin{frame}[fragile]{Definición de \texttt{esNil}}
    \begin{exampleblock}{Parte II}
        Definir las funciones \code{esNil}, \code{altura}, \code{cantNodos} (para \code{esNil} puede utilizarse \code{case} en lugar de \code{foldAB}).
    \end{exampleblock}

    \pause

    Definición de \code{esNil}:

    \pause

    \begin{minted}[escapeinside=||]{haskell}
        esNil :: (AB a) -> Bool|\pause|
        esNil ab = case ab of|\pause|
            Nil       -> True|\pause|
            Bin _ _ _ -> False
    \end{minted}

    \pause

    Definición alternativa usando \code{foldAB}:
    \begin{minted}[escapeinside=||]{haskell}
        esNil :: (AB a) -> Bool|\pause|
        esNil = foldAB f z|\pause|
            where
                z = True|\pause|
                f _ _ _ = False|\pause|
    \end{minted}
\end{frame}

\begin{frame}[fragile]{Definición de \texttt{altura}}
    La \textbf{altura} de un árbol es la longitud de la rama más larga.

    \pause

    Definición de \code{altura}:

    \begin{minted}[escapeinside=||]{haskell}
        altura :: (AB a) -> Int|\pause|
        altura = foldAB f z|\pause|
            where
                z = 0|\pause|
                f rIzq _ rDer = 1 + max rIzq rDer
    \end{minted}
\end{frame}

\begin{frame}[fragile]{Definición de \texttt{cantNodos}}
    Definición de \code{cantNodos}:

    \pause

    \begin{minted}[escapeinside=||]{haskell}
        cantNodos :: (AB a) -> Int|\pause|
        cantNodos = foldAB|\pause| (\rIzq _ rDer -> 1 + rIzq + rDer)|\pause| 0
    \end{minted}
\end{frame}


\section{Parte III}

\begin{frame}{Parte III -- Consigna}
    \begin{exampleblock}{Parte III}
        Definir la función \code{mejorSegún :: (a -> a -> Bool) -> AB a -> a} para árboles. Se recomienda definir una función auxiliar para comparar la raíz con un posible resultado de la recursión para un árbol que puede o no ser \code{Nil}.
    \end{exampleblock}
\end{frame}

\begin{frame}[fragile]{Función Auxiliar}
    Primedro, implementamos la función auxiliar que indica la consigna, que compara la raíz con un posible resultado de la recursión. Para modelar un ``posible resultado'', usamos el tipo \code{Maybe}:

    \pause

    \begin{minted}[escapeinside=\#\#]{haskell}
        maxSegunM :: (a -> a -> Bool) -> a -> Maybe a -> a#\pause#
        maxSegunM cmp x Nothing = x#\pause#
        maxSegunM cmp x (Just y)#\pause#
            | x `cmp` y = x
            | otherwise = y
    \end{minted}
\end{frame}

\begin{frame}[fragile]{Definición de \texttt{mejorSegun}}
    Ahora podemos definir \code{mejorSegun}. La función será \alert{parcial}, ya que no tenemos forma de producir un valor de tipo \code{a} cuando la entrada es \code{Nil}.

    \pause

    \begin{minted}[fontsize=\small,escapeinside=||]{haskell}
        mejorSegun :: (a -> a -> Bool) -> AB a -> a|\pause|
        mejorSegun cmp ab = fromJust $ foldAB maxSegun3 Nothing ab|\pause|
            where
                maxSegun3 :: Maybe a -> a -> Maybe a -> Maybe a|\pause|
                maxSegun3 mIzq raiz mDer =
                    Just $ maxSegunM (maxSegunM raiz mIzq) mDer
    \end{minted}
\end{frame}


\input{sections/parte4.tex}

\input{sections/parte5.tex}

\end{document}
