\section{Parte II}

\begin{frame}[fragile]{Definición de \texttt{esNil}}
    \begin{exampleblock}{Parte II}
        Definir las funciones \code{esNil}, \code{altura}, \code{cantNodos} (para \code{esNil} puede utilizarse \code{case} en lugar de \code{foldAB}).
    \end{exampleblock}

    \pause

    Definición de \code{esNil}:

    \pause

    \begin{minted}[escapeinside=||]{haskell}
        esNil :: (AB a) -> Bool|\pause|
        esNil ab = case ab of|\pause|
            Nil       -> True|\pause|
            Bin _ _ _ -> False
    \end{minted}

    \pause

    Definición alternativa usando \code{foldAB}:
    \begin{minted}[escapeinside=||]{haskell}
        esNil :: (AB a) -> Bool|\pause|
        esNil = foldAB f z|\pause|
            where
                z = True|\pause|
                f _ _ _ = False|\pause|
    \end{minted}
\end{frame}

\begin{frame}[fragile]{Definición de \texttt{altura}}
    La \textbf{altura} de un árbol es la longitud de la rama más larga.

    \pause

    Definición de \code{altura}:

    \begin{minted}[escapeinside=||]{haskell}
        altura :: (AB a) -> Int|\pause|
        altura = foldAB f z|\pause|
            where
                z = 0|\pause|
                f rIzq _ rDer = 1 + max rIzq rDer
    \end{minted}
\end{frame}

\begin{frame}[fragile]{Definición de \texttt{cantNodos}}
    Definición de \code{cantNodos}:

    \pause

    \begin{minted}[escapeinside=||]{haskell}
        cantNodos :: (AB a) -> Int|\pause|
        cantNodos = foldAB|\pause| (\rIzq _ rDer -> 1 + rIzq + rDer)|\pause| 0
    \end{minted}
\end{frame}
